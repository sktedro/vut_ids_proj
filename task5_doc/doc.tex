%! Authors: Patrik Skaloš, Dziyana Khrystsiuk

% Preamble
\documentclass[a4paper]{article}


% Packages
\usepackage[utf8]{inputenc}
\usepackage[english]{babel}
\usepackage[T1]{fontenc}
\usepackage[total={18cm, 25cm}]{geometry}
% \usepackage{amsmath, amssymb, amsthm}
% \usepackage[unicode]{hyperref}
% \usepackage{graphicx}


% Title
\title{Dokumentace popisující finální schéma databáze}
\author{Dziyana Khrystsiuk (xkhrys00) \\ Patrik Skaloš (xskalo01)}
\date{}


% Document
\begin{document}

  \maketitle


  \section{Triggre}

  \subsection{Trigger 1}
  Rozhodli sme sa pridať takú funkcionalitu, že po vytvorení objednávky (ktorá
  už nemôže byť zrušená) sa zo skladu odčíta taký počet surovín, aký je potrebný
  na upečenie objednaného pečiva. Ak si teda napríklad zákazník objedná jeden 
  chlieb, na ktorý je okrem iného potreba 500 gramov múky, z aktuálneho množstva
  múky na sklade sa 500 gramov odčíta. To sa ale nedeje pre predmety, ktoré môžu
  byť v pečive zapečené, keďže tie sa nakupujú až po prijatí objednávky.

  \subsection{Trigger 2}
  Druhý trigger sme naprogramovať nestihli a predpokladali sme, že to nie je
  také dôležité, keď sme už tým prvým ukázali, že tomu rozumieme.


  \section{Procedúry}

  \subsection{Procedúra 1: \textit{SIZE\_CHK}}
  Naša prvá procedúra (funkcia) je zodpovedná za overenie, či sa predmet naviac
  (ktorý má byť zapečený do pečiva) svojimi rozmermi do pečiva vojde. Napríklad,
  ak predávame chlieb s rozmermi 200x200x200mm a zákazník si chce objednať 
  šrubovák s rozmermi 300x50x50mm, procedúra vráti hodnotu \verb|1| na miesto
  hodnoty \verb|0|. Vďaka tejto funkcii užívateľ nemôže vytvoriť objednávku s
  takými pármi pečív a predmetov, v ktorých sa predmet nevojde do pečiva celý.

  \subsection{Procedúra 2: \textit{NUM\_OF\_HOURS}}
  Naša druhá funkcia počíta a vracia počet hodín, ktoré strážnik odpracoval
  do istého dátumu a času od začiatku toho mesiaca. Funkcia používa kurzor 
  \textit{hours}, v ktorom sú časy začiatku a konca smeny daného strážnika v 
  daný mesiac. Funkcia taktiež používa premennú \textit{shift\_info}, ktorá je
  takého dátového typu, ako riadky vrátené kurzorom (pomocou \textit{\%ROWTYPE}).


  \section{Index}
  Index sme nedokázali naprogramovať. V nasledujúcej sekcii spomenieme náš nápad
  na indexáciu.


  \section{EXPLAIN PLAN}
  EXPLAIN PLAN exportovaný pre príkaz SELECT, ktorý vráti počty väzňov pre každú
  väznicu v databáze:

  \begin{verbatim}
    Plan hash value: 2904766977

    -----------------------------------------------------------------------------------------------
    | Id  | Operation                    | Name           | Rows  | Bytes | Cost (%CPU)| Time     |
    -----------------------------------------------------------------------------------------------
    |   0 | SELECT STATEMENT             |                |     5 |   600 |     5  (20)| 00:00:01 |
    |   1 |  NESTED LOOPS                |                |     5 |   600 |     5  (20)| 00:00:01 |
    |   2 |   NESTED LOOPS               |                |     5 |   600 |     5  (20)| 00:00:01 |
    |   3 |    VIEW                      | VW_GBF_7       |     5 |   130 |     4  (25)| 00:00:01 |
    |   4 |     HASH GROUP BY            |                |     5 |    65 |     4  (25)| 00:00:01 |
    |   5 |      TABLE ACCESS FULL       | customer       |     5 |    65 |     3   (0)| 00:00:01 |
    |*  6 |    INDEX UNIQUE SCAN         | SYS_C002006943 |     1 |       |     0   (0)| 00:00:01 |
    |   7 |   TABLE ACCESS BY INDEX ROWID| prison         |     1 |    94 |     1   (0)| 00:00:01 |
    -----------------------------------------------------------------------------------------------

    Predicate Information (identified by operation id):
    ---------------------------------------------------

    "   6 - access(""P"".""prison_id""=""ITEM_1"")"

    Note
    -----
      - dynamic statistics used: dynamic sampling (level=2)
      - this is an adaptive plan
  \end{verbatim}

  Žiaden ďalší ESCAPE PLAN sme negenerovali, keďže sa nám nepodarilo vytvoriť 
  žiaden index. Našim plánom však bolo indexovať podľa identifikačných čísel
  väzníc.


  \section{Definícia prístupových práv}
  Patrikovi Skalošovi sme dali prístupové práva ako väzňovi (zákazníkovi):

  Môže čítať len jeho vlastné objednávky a prehliadať ponúkané pečivo a predmety
  s tým, že sa mu zobrazia iba informácie dôležité pre zákazníkov.


  \section{Materializovaný pohľad}
  Náš materializovaný pohľad vytvorený pre Patrika Skaloša ako zákazníka 
  Richarda mu poskytuje informácie o všetkých jeho objednávkach a o ničom inom.
  Stĺpce tohto pohľadu teda sú: Dátum a čas objednávky, očakávaný (ak je v
  minulosti, tak pravdivý) dátum doručenia, spôsob doručenia a obsah objednávky.


  \vspace{1in}

  PS: Server bol počas vývoja extrémne pomalý, bežne trvalo 5 minút na vykonanie
  jedného jednoduchého príkazu INSERT či SELECT, čo značne sťažilo možnosť
  testovať túto časť projektu a preto sme jej väčšinu ani neotestovali a
  nedokončili.

\end{document}
